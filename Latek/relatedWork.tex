\subsection{Existing Connections}
This work goes back to the familiarity of both Abe and I with materials science. We both enrolled in materials science freshman year and are excited to take a different approach to the investigation of material properties. Additionally we both have a huge respect for the work that Spiderman does to keep this world safe and feel that we truly owe it to him to ensure that the world know about the legitimacy of his abilities.

\subsection{External Knowledge}
2. Citations:

\begin{enumerate}

\item I. Agnarsson, M. Kuntner, and T. A. Blackledge, “Bioprospecting Finds the Toughest Biological Material: Extraordinary Silk from a Giant Riverine Orb Spider,” PLoS One, vol. 5, no. 9, Sep. 2010.

This paper is about a particularly tough spider silk - the toughest ever found. It's not the most elastic, but it can take way way way more strain. Additionally, it contains a table of data on many spider silk's elasticity.

\item J. Gerstmayr, H. Irschik, Vibrations of the elasto-plastic pendulum, International Journal of Non-Linear Mechanics, Volume 38, Issue 1, January 2003, Pages 111-122, ISSN 0020-7462, 10.1016/S0020-7462(01)00052-X.

This paper is an approach to modeling a pendulum with elastic and plastic properties. While we may not ever include plasticity in our model, this is a good framework if we do

\item A. M. F. Moore and K. Tran, “Material properties of cobweb silk from the black widow spider Latrodectus hesperus,” International Journal of Biological Macromolecules, vol. 24, no. 2–3, pp. 277–282, Mar. 1999.

Another paper on the material properties of spider silk - can;t have enough of those!

\item J.M. Tuwankotta, G.R.W. Quispel, Geometric numerical integration applied to the elastic pendulum at higher-order resonance, Journal of Computational and Applied Mathematics, Volume 154, Issue 1, 1 May 2003, Pages 229-242, ISSN 0377-0427, 10.1016/S0377-0427(02)00825-7.

A numerical method for dealing with an elastic 3-d pendulum. While it's more focused on resonance than us, it's a good way of familiarizing ourselves with techniques used to deal with these pendulums.

\item F. Vollrath, “Strength and structure of spiders’ silks,” Reviews in Molecular Biotechnology, vol. 74, no. 2, pp. 67–83, Aug. 2000.

An paper describing the particulars of the strength of spider silk. key for our work - the mechanics of spider silk are at the heart of our model.

\end{enumerate}