\subsection{Overall Goal}
We want to determine the feasibility of Spiderman's notorious web-slinging abilities. We would like to examine this both from a materials perspective, considering the strength of the web he uses, and from the physical perspective of swinging from an elastic pendulum while firing mass in the opposite direction to motion. We also want to determine the ideal pendulum or web-slinging strategy for spiderman to travel at top speed through a city street.

\subsection{Specific objectives}
We want to accurately model spiderman's web-slinging through an idealized environment consisting of two infinite walls on either side of a street with a fixed width and infinite length. Initially we will consider the webs to be simple springs and seek mainly to discover what the material limitations on his web are as far as strength, spring constant, and strength. We would then like to do a regression testing using a bevy of assorted controls algorithms to attempt to arrive at an ideal strategy for traversal.

\subsection{Educational significance}
Abe: Very interested in the materials science part of this assignment, as well as generally developing a better understanding of the modelling of physical system in a rigorous way. Currently enrolled in Stuff of History, and wants to develop a better understanding of tensile and stress properties.

Slater: Very interesting the machine learning portion of the regression testing, as well as the controls algorithms that tie directly into his research on combined fuel cycles. Also very interested in materials science and failure analysis. Took material science freshman year and NINJA'd stuff second semester.

\subsection{Intellectual Impact}
This is a system that would seem simple at first, but upon closer inspection it quickly becomes an intricate mechanical problem. This problem also is a great illustration of a number of simple systems coming together to form a system that quickly becomes extremely complex. Additionally, it is a problem that has a simple to asses entry point that also allows for an extension to almost arbitrary complexity with interesting problems arising at every point along the way.

\subsection{Broader Impacts}
The iconic web-slinging superhero has been seen by innumerable people world round, but as of yet there has been no real quantitative investigation into his abilities. The one attempt that was made by Emory professor Skip Garibaldi has received a huge amount of press coverage. Additionally, this could have potentially large rammifications within the physics education space, encouraging students at the level of high school and above to thoroughly investigate higher level physics as a tool for the investigation of their childhood superheroes.
