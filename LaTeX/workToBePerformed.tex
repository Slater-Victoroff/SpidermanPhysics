\subsection{Pre-existing Work}
As far as pre-existing work goes we originally investigated the work done by Emory professor Skip Garibaldi, but upon further contact it appears that his work will not be extremely useful to us. Instead we're looking to base our work on the previous work done on elastic pendulums by A.H.P. van der Burgh(1975) and Vetyukov(2004). Additionally we will base our treatment of spiderweb's material properties on Vollrath(2000). These are appropriate resource because not only to they directly deal with the technical areas in our project, but the additionally are very comprehensive and should give an overview wide enough to properly get us up to speed on the current state of this branch of physics.

\subsection{Pre-Existing Data}
For pre-existing data we intend to use the data sets on spider silk present in both Vollrath(2000) and Agnarsson(2010). As far as the data for the elastic pendulum goes we will attempt to put together a basic experimental setup to ensure that we have the basic physics accurately modelled. Past this basic level of verification however it will be difficult to compare our model with existing data beyond our own intuitions due to the novel nature of the work. We will compare our results to the countless representations of our work throughout popular media, however these representations do not make for an  accurate representation of reality.

\subsection{Available Software}
We are planning on using numpy for the simulations themselves as well as openopt for the optimization portion of our project. These are python libraries designed for fast numerical calculation and fast nonlinear optimization respectively. We will also likely use vpython for visualizations of our simulations.

\subsection{Experimental Methods}
See Pre-Existing Data

\subsection{Timeline}
Before the official start of the project (April 19):
       We will familiarize ourselves with models of elastic pendulums.
   During Week 1 of the project:
   
       Over the weekend (April 19-April 23) we will approach an analytical and a numerical solution to the problem of a spherical spring pendulum, likely only finding the numerical
       During the week (April 23-April 26) we will work on a full simulation of spiderman, with the goal of being able to simulate Spiderman's web-slinging in different scenarios.
   During Week 2 of the project:
       Over the weekend (April 16-April 30) we will approach the problem of optimizing
       	  Spiderman's webslinging, using openopt and simple AIs. We don't expect to finish this part over the weekend, but we will make a valiant effort
       On the last day of class (April 30) we will continue working on our simulations and decide how much work we will continue to do
       Between the last day and the presentation (April 30-May 9) we will put together a paper and presentation and finish any optimizations we've deemed acceptable.
